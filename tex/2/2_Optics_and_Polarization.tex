\documentclass[twocolumn,draft]{article}
\usepackage{savetrees}
% a more conservative option:
% \documentclass[]{article}
% \usepackage{fullpage}

\usepackage[utf8]{inputenc}
\usepackage[T1]{fontenc}
\usepackage{microtype}

\usepackage{amssymb}
\usepackage{amsfonts}
\usepackage{amsmath}
\usepackage{amsthm}

\usepackage{todonotes}
\usepackage{graphicx}
\usepackage{showlabels}

\usepackage[english]{babel}
\usepackage{blindtext}
% \Blindtext to use.

\usepackage{graphicx}

\RequirePackage[l2tabu, orthodox]{nag}
\usepackage[all,warning]{onlyamsmath}

\usepackage{hyperref}

% blue text for questions
\newcommand{\bt}[1]{\textcolor{blue}{#1}}

\title{PY 202, 4/3/17}
\author{John M. Lynch\footnote{Undergraduate ECE/Physics, NCSU, Raleigh, NC 27705. E-Mail: \texttt{jmlynch3@ncsu.edu}}}
\date{\today}

\begin{document}
  \maketitle
\section*{In-Class Quiz}

Explain, in your own words (sketches allowed), the following terms:

\begin{enumerate}
	\item Reflection
	\item Refraction
	\item Dispersion
\end{enumerate}

\section*{In-Class Quiz Answers}

\begin{enumerate}
	\item \underline{Reflection}: when no incident light on the other side of a material. Could also
					  say that the incident angle $\theta_{i}$ and the reflective angle $\theta_{r}$
					  are equal:
					  \begin{equation}
					  	\theta_{i} = \theta_{r}
					  \end{equation}
					  For example, when using a mirror.
	\item \underline{Refraction}: the bending of light when it moves from one medium to another.
								  Involves two transparent materials with different indices
								  of refraction. \emph{The bending of light at the interface}.
								  Governed by Snell's Law:
								  \begin{equation}
								  	n_{1}\sin{\theta_{1}} = n_{2}\sin{\theta_{2}}
								  \end{equation}
	\item \underline{Dispersion}: when the angle of reflection is different for each wavelength
		 						  of input light. When the index of refraction of a material
								  depends on the wavelength (e.g. with air and a glass prism).
								  Examples: rainbow, prism
	\item \underline{Diffraction}: (Bonus) when light enters a curved surface
\end{enumerate}

\section*{Optical Phenomena}

\subsection*{Reflection}
\begin{enumerate}
	\item $\theta_{i} = \theta_{r}$
	\item if you have a perfectly flat mirror, parallel incident light will reflect in parallel
		  (\emph{Specular reflection})
	\item if your surface is \emph{not} flat, parallel incident light will \emph{not} reflect in
		  parallel (\emph{Diffuse reflection})
\end{enumerate}

\subsection*{Refraction}
\begin{enumerate}
	\item with transparent materials of different indices of refraction, initial material of
	index $n_{1}$ and other medium of index $n_{2}$, 
		\[
			n_{1}\sin{\theta_{1}} = n_{2}\sin{\theta_{2}}
		\]
	\item $n_{1}>n_{2}$: the larger $n_{2}$ is, the smaller $\theta_{2}$ becomes
	\item can infer relative index magnitudes by comparing angle of incidence to angle of
	refraction
\end{enumerate}

\subsection*{Critical Angle}
	\begin{equation*}
		\boxed{\sin{\theta_{c}} = \frac{n_{2}}{n_{1}}}
	\end{equation*}
\noindent critical angle is dividing line between two different behaviors:
	\begin{enumerate}
		\item if $\theta < \theta_{c}$, reflected \emph{\&} refracted light
		\item if $\theta = \theta_{c}$, reflected $\cong$ refracted: exactly along the interface
			  between the two materials
		\item if $\theta > \theta_{c}$, no reflected light (\emph{total internal reflection})
	\end{enumerate}

% TODO: correctly implement PNG imports
\subsection*{Water Tank Problem}
\bt{discuss with Dr. Frohlic!}

\section*{Polarization}
Assume wave traveling to right:
%\includegraphics{"./waves"}
%\includegraphics{"second".png}

\section*{Propagation}
% TODO: Turn the below into a table
% TODO: import the board with a filter applied to it
% TODO: consider a script boardImport.py
% 			- sorta here: http://registry.gimp.org/node/19822

\section*{WTP Image}
%\includegraphics[scale=0.1]{"wtp".png}

\section*{Quiz Topic for Friday}
Reflection, Refraction, Critical Angle





  
\end{document}
