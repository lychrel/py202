\documentclass[twocolumn,draft]{article}
\usepackage{savetrees}
% a more conservative option:
% \documentclass[]{article}
% \usepackage{fullpage}

\usepackage[utf8]{inputenc}
\usepackage[T1]{fontenc}
\usepackage{microtype}

\usepackage{amssymb}
\usepackage{amsfonts}
\usepackage{amsmath}
\usepackage{amsthm}

\usepackage{todonotes}
\usepackage{graphicx}
\usepackage{showlabels}

\usepackage[english]{babel}
\usepackage{blindtext}
% \Blindtext to use.

\usepackage{graphicx}

\RequirePackage[l2tabu, orthodox]{nag}
\usepackage[all,warning]{onlyamsmath}

\usepackage{hyperref}

% blue text for questions
\newcommand{\bt}[1]{\textcolor{blue}{#1}}

\title{PY 202, 4/7/17}
\author{John M. Lynch\footnote{Undergraduate ECE/Physics, NCSU, Raleigh, NC 27705. E-Mail: \texttt{jmlynch3@ncsu.edu}}}
\date{\today}

\begin{document}
  \maketitle
\section*{HW Note: Fishing}
% See image
\section*{In-Class Quiz}
\begin{enumerate}
	\item 
		light passes into two different media, deflects at two different angles (one smaller, one
		larger). In which media is the light traveling \emph{faster}?
			\begin{enumerate}
				\item the two things you consider are
				\begin{equation*}
					n_{1}\sin{\theta_{1}} = n_{2}\sin{\theta_{2}}
				\end{equation*}
				so
				\begin{equation*}
					\theta_{1} > \theta_{2} \qquad\Longrightarrow\qquad n_{1} < n_{2}
				\end{equation*}
				and
				\begin{equation*}
					n = \frac{c}{v}
				\end{equation*}
				so
				\begin{equation*}
					n_{1} = \frac{c}{v_{1}} < n_{2}  \frac{c}{c_{2}}
				\end{equation*}
				and
				\begin{equation*}
					v_{1} > v_{2}
				\end{equation*}
			\end{enumerate}
			
	\item to shoot a fish, where should you aim?
		\begin{enumerate}
			\item aim slightly below. The light, entering the liquid of higher index of refraction
				than air, will deflect from the vertical, implying that the fish is lower. Your
				brain just extrapolates it to be where the light would go un-refracted.
		\end{enumerate}
		
	\item Now you're shooting a laser gun. Where do you aim w/r/t where you see the fish?
		\begin{enumerate}
			\item aim it at the real image. It'll deflect the same way the visible light does
				and hits the fish. You can assume the laser's wavelength is similar to visible
				light, so it doesn't affect index of refraction $n(\lambda)$. And even if a 
				difference in indices of refraction did exist, it wouldn't be extreme enough to
				miss the fish. 
		\end{enumerate}
	\item \underline{quiz grade:} $3/3 = 100$\%
\end{enumerate}

Moral of story: brains can't infer the correct angle, which often leads to misunderstandings.

\subsection*{Examples}
\begin{enumerate}
	\item 
		Light travels through a material with a speed 0.85c. What is teh critical angle for total
		internal
		reflection of a light ray at the interface between the material and a vacuum?

		Well, $\sin{\theta_{2}}=90~deg$, so

		\begin{equation}
			\sin{\theta_{L}} = \frac{n_{2}}{n_{1}}
		\end{equation}

		with

		\begin{align*}
			n &= \frac{c}{v} \\
			v_{1} &= c \\
			v_{2} &= 0.85c
		\end{align*}

		use to solve.
	
	\item
		ray of light with two wavelengths hits piece of acrylic at angle 30 from surface (red
		herring:
		that isn't the angle of incidence!). What's the angular separation between red and blue
		light
		rays that pass through the acrylic? (Second distraction: the index of refraction of acrylic
		doesn't matter.)
		
		Just calculate the two angles of refraction and calculate their difference. To tell which
		one is which: shorter wavelength means smaller angle. Angle of incidence is found by
			\begin{equation*}
				\theta_{i} = 90 - 30 = 60 ~deg
			\end{equation*}
			
	\item light ray passing through two media; infer relative magnitudes of indices of refraction
		from the relative angle sizes (i.e. it deflects more coming out of the material than it
		does going in).
			\begin{align*}
				n_{2} &> n_{1} \\
				n_{3} &< n_{2} \\
				n_{1} &> n_{3}
			\end{align*}
		Everything is measured with respect to a surface normal axis/vector; that's really important
		to remember.
			
\end{enumerate}

\section*{Polarization}

Important thing to remember is

\begin{equation*}
	I = I_{0}\cos^{2}{\theta}
\end{equation*}

where $I_{0}$ is the intensity pre-hit and $\theta$ is the angle w/r/t the previous polarizer

\section*{Line Spectra and Atoms}

  
\end{document}
