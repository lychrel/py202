\documentclass[twocolumn,draft]{article}
\usepackage{savetrees}
% a more conservative option:
% \documentclass[]{article}
% \usepackage{fullpage}

\usepackage[utf8]{inputenc}
\usepackage[T1]{fontenc}
\usepackage{microtype}

\usepackage{amssymb}
\usepackage{amsfonts}
\usepackage{amsmath}
\usepackage{amsthm}

\usepackage{todonotes}
\usepackage{graphicx}
\usepackage{showlabels}

\usepackage[english]{babel}
\usepackage{blindtext}
% \Blindtext to use.

\RequirePackage[l2tabu, orthodox]{nag}
\usepackage[all,warning]{onlyamsmath}

\usepackage{hyperref}

% blue text for questions
\newcommand{\bt}[1]{\textcolor{blue}{#1}}

\title{PY 202, 4/3/17}
\author{John M. Lynch\footnote{Undergraduate ECE/Physics, NCSU, Raleigh, NC 27705. E-Mail: \texttt{jmlynch3@ncsu.edu}}}
\date{\today}

\begin{document}
  \maketitle
  % major enum
  \section{Qualitative}
  \begin{enumerate}
  	\item laser-box and mirrors demonstration
		\begin{enumerate}
			\item consider, effectively, all the light
		in parallel
			\item convex mirror: they diverge out
			\item concave mirror: they focus in
		\end{enumerate}
	\item deflection of light entering water
		\begin{enumerate}
		\item moving from any material to any other
		material will change direction of light
		\item \emph{total internal reflection}
			\begin{enumerate}
			\item determined by \emph{index of refraction} and Snell's law:
			\begin{equation}
				n_{1}\sin{(\theta_{1})} = n_{2}\sin{(\theta_{2})}
			\end{equation}
			\begin{enumerate}
				\item $n$'s are indices of refraction
			\end{enumerate}
		\end{enumerate}
		\end{enumerate}
	\item laser light bending with water: principle
	behind the use of \emph{fiberoptics}
		\begin{enumerate}
		\item Hits interface at an angle such that it can't escape.
		\item if you never get any component that goes out, you always reflect everything on the inside.
		\item \bt{ask about this!}
	\end{enumerate}
	\item Air's index of refraction changes with
	temperature
		\begin{enumerate}
			\item example: hot air rising from the
			ground makes objects seem to ``flip''
		\end{enumerate}
	\item bottle of water
		\begin{enumerate}
			\item water's full of beads, but, when
			together in jar, they look like water.
			Why?
				\begin{enumerate}
					\item because they have the 
					\emph{same index of refraction
						as water}!
				\end{enumerate}
		\end{enumerate}
  \end{enumerate}
  
  \section{Quantitative}
  \begin{enumerate}
	\item Index of refraction:
		\begin{equation}
			n = \frac{c}{v}
		\end{equation}
		\begin{enumerate}
		  	\item $n_{air} = 1$
			\item $n > 1$ (all others)
			\item $v \leq c$
				\begin{enumerate}
					\item $v=$ speed of light in
					material
				\end{enumerate}
		\end{enumerate}
	\item wavelength in material
		\begin{equation}
			\lambda_{n} = \frac{\lambda}{n}
						= \frac{\lambda}{\frac{c}{v}}
						= \frac{v}{c}\lambda
		\end{equation}
	\item angle of incidence measured from source to
	\emph{vertical}
	\item angle of deflection is measured from
	opposite vertical (i.e. same axis)
		\begin{enumerate}
			\item \bt{add pic later!}
		\end{enumerate}
	\item flat mirror gives perfect reflection:
		\begin{equation}
			\theta_{1} = \theta_{2}
		\end{equation}
  \end{enumerate}
  
  \section{Homework}
  \noindent Read Moodle
  
\end{document}
